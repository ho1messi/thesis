% !Mode:: "TeX:UTF-8"
\documentclass[capcenterlast=true,subcapcenterlast=true,openright=true,absupper=true,fontset=windowsnew,type=bachelor]{hithesis}
% 此处选项中不要有空格
%%%%%%%%%%%%%%%%%%%%%%%%%%%%%%%%%%%%%%%%%%%%%%%%%%%%%%%%%%%%%%%%%%%%%%%%%%%%%%%%
% 必填选项
% type=doctor|master|bachelor
%%%%%%%%%%%%%%%%%%%%%%%%%%%%%%%%%%%%%%%%%%%%%%%%%%%%%%%%%%%%%%%%%%%%%%%%%%%%%%%%
% 选填选项(选填选项的缺省值已经尽可能满足了大多数需求,除非明确知道自己有什么
% 需求)
% glue=true|false
% 	含义:由于我工规范中要求字体行距在一个闭区间内,这个选项为true表示tex自
% 	动选择,为false表示区间内一个最接近版心要求行数的要求的默认值,缺省值为
% 	false。
% tocfour=true|false
% 	含义:是否添加第四级目录,只对本科文科个别要求四级目录有效,缺省值为
% 	false
% fontset=siyuan|windowsnew|windowsold
% 	含义:注意这个选项视为了解决特殊问题而设置,比如用有些发行版本的linux排
% 	版时可能(大多数发行版不会)会遇到的字体无法载入的问题,或者字体载入之
% 	后出现无法复制的问题以及想要解决排版如 biang biang 面的 biang 这类中易
% 	宋体无法识别的汉字的问题。没有特殊的需要不推荐使用这个选项。
%
% 	如果是安装了 windowns 字体的 linux 系统,可以填写windowsnew(win vista
% 	以后 的字体)或 windowsold(vista 以前)或者想用思源宋体并且是已经安装
% 	了思源宋体的任何系统,填写siyuan选项。缺省值为空,自动识别系统并匹配字体
% 	。模板版中给出的思源字体定义文件定义的思源字体的版本是Adobe版,其他字体
% 	是windowsnew字体。
% tocblank=true|false
% 	含义:目录中第一章之前,是否加一行空白。缺省值为true。
% chapterhang=true|false
% 	含义:目录的章标题是否悬挂居中,规范中要求章标题少于15字,所以这个选项
% 	有无没什么用,除了特殊需求。缺省值为true。
% fulltime=true|false
% 	含义:是否全日制,缺省值为true。非全日制如同等学力等,要在cover中设置类
% 	型,封面中不同格式
% subtitle=true|false
% 	含义:论文题目是否含有副标题,缺省值为false,如果有要在cover中设置副标
% 	题内容,封面中显示。
% newgeometry=true|false
% 	含义:规范中的自相矛盾之处,版芯是否包含页眉页脚,旧方法是按照包含页眉
% 	页脚来设置,缺省值为false,即旧方法。
% debug=true|false
% 	含义:是否显示版芯框和行号,用来调试。默认否。
% openright=true|false
% 	含义:博士论文是否要求章节首页必须在奇数页,此选项不在规范要求中,按个
% 	人喜好自行决定。 默认否。注意,窝工的默认情况是打印版博士论文要求右翻页
% 	,电子版要求非右翻页且无空白页。如果想DIY(或身不由己DIY)在什么地方右
% 	翻页,将这个选项设置为false,然后在目标位置添加`\cleardoublepage`命令即
% 	可。
% capcenterlast=true|false
% 	含义:图题、表题最后一行是否居中对齐(我工规范要求居中,但不要求居中对
% 	齐),此选项不在规范要求中,按个人喜好自行决定。默认否。
% subcapcenterlast=true|false
% 	含义:子图图题最后一行是否居中对齐(我工规范要求居中,但不要求居中对齐
% 	),此选项不在规范要求中,按个人喜好自行决定。默认否。
% absupper=true|false
%       含义:中文目录中的英文索引在中文目录中的大小写样式歧义,在规范中要求首
%       字母大写,在work样例中是全大写。该选项控制是否全大写。默认否。
%%%%%%%%%%%%%%%%%%%%%%%%%%%%%%%%%%%%%%%%%%%%%%%%%%%%%%%%%%%%%%%%%%%%%%%%%%%%%%%%

\usepackage{hithesis}
\graphicspath{{figures/}}

\begin{document}

\frontmatter
% !Mode:: "TeX:UTF-8"

\hitsetup{
  %******************************
  % 注意:
  %   1. 配置里面不要出现空行
  %   2. 不需要的配置信息可以删除
  %******************************
  %
  %=====
  % 秘级
  %=====
  statesecrets={公开},
  natclassifiedindex={TM301.2},
  intclassifiedindex={62-5},
  %
  %=========
  % 中文信息
  %=========
  ctitleone={基于图像识别技术的},%本科生封面使用
  ctitletwo={导游软件},%本科生封面使用
  ctitlecover={基于图像识别技术的导游软件},%放在封面中使用,自由断行
  ctitle={基于图像识别技术的导游软件},%放在原创性声明中使用
  csubtitle={一条副标题}, %一般情况没有,可以注释掉
  cxueke={工学},
  csubject={计算机科学与技术},
  caffil={计算机科学与技术学院},
  cauthor={石宇泽},
  csupervisor={伯彭波},
  % cassosupervisor={某某某教授}, % 副指导老师
  % ccosupervisor={某某某教授}, % 联合指导老师
  % 日期自动使用当前时间,若需指定按如下方式修改:
  cdate={2018年6月12日},
  cstudentid={140410229},
  cstudenttype={同等学力人员}, %非全日制教育申请学位者
  %(同等学力人员)、(工程硕士)、(工商管理硕士)、
  %(高级管理人员工商管理硕士)、(公共管理硕士)、(中职教师)、(高校教师)等
  %
  %
  %=========
  % 英文信息
  %=========
  etitle={Research on key technologies of partial porous externally pressurized gas bearing},
  esubtitle={This is the sub title},
  exueke={Engineering},
  esubject={Computer Science and Technology},
  eaffil={\emultiline[t]{School of Mechatronics Engineering \\ Mechatronics Engineering}},
  eauthor={Yu Dongmei},
  esupervisor={Professor XXX},
  eassosupervisor={XXX},
  % 日期自动生成,若需指定按如下方式修改:
  edate={December, 2017},
  estudenttype={Master of Art},
  %
  % 关键词用“英文逗号”分割
  ckeywords={\TeX, \LaTeX, CJK, 嗨!, thesis},
  ekeywords={\TeX, \LaTeX, CJK, template, thesis},
}

\begin{cabstract}

摘要的字数(以汉字计),硕士学位论文一般为500 $\sim$ 1000字,博士学位论文为1000 $\sim$ 2000字,
均以能将规定内容阐述清楚为原则,文字要精练,段落衔接要流畅。摘要页不需写出论文题目。
英文摘要与中文摘要的内容应完全一致,在语法、用词上应准确无误,语言简练通顺。
留学生的英文版博士学位论文中应有不少于3000字的“详细中文摘要”。

  关键词是为了文献标引工作、用以表示全文主要内容信息的单词或术语。关键词不超过 5
  个,每个关键词中间用分号分隔。(模板作者注:关键词分隔符不用考虑,模板会自动处
  理。英文关键词同理。)
\end{cabstract}

\begin{eabstract}
   An abstract of a dissertation is a summary and extraction of research work
   and contributions. Included in an abstract should be description of research
   topic and research objective, brief introduction to methodology and research
   process, and summarization of conclusion and contributions of the
   research. An abstract should be characterized by independence and clarity and
   carry identical information with the dissertation. It should be such that the
   general idea and major contributions of the dissertation are conveyed without
   reading the dissertation.

   An abstract should be concise and to the point. It is a misunderstanding to
   make an abstract an outline of the dissertation and words ``the first
   chapter'', ``the second chapter'' and the like should be avoided in the
   abstract.

   Key words are terms used in a dissertation for indexing, reflecting core
   information of the dissertation. An abstract may contain a maximum of 5 key
   words, with semi-colons used in between to separate one another.
\end{eabstract}
 % 封面
\makecover
% \input{front/denotation}%物理量名称表,符合规范为主,有要求添加
%\cleardoublepage  自定义在什么位置进行右翻页
\tableofcontents    % 中文目录
%\cleardoublepage  自定义在什么位置进行右翻页
%\tableofengcontents % 英文目录,硕本不要求

\mainmatter
%\linenumbers %debug 选项
%\layout %debug 选项
%\floatdiagram %debug 选项
%\begin{figure} %debug 选项
%\currentfloat %debug 选项
%\tryintextsep{\intextsep} %debug 选项
%\trytopfigrule{0.5pt} %debug 选项
%\trybotfigrule{1pt} %debug 选项
%\setlayoutscale{0.9} %debug 选项
%\floatdesign %debug 选项
%\caption{Float layout with rules}\label{fig:fludf} %debug 选项
%\end{figure} %debug 选项
%\include{body/introduction}
\chapter{绪论}
	
	\section{课题背景及研究的目的和意义}
	
		\subsection{课题背景}
		
			随着中国经济的发展,人们的消费水平越来越高,旅游越来越成为一种普遍的爱好。每逢节假日,各地景区都会迎来许多游客。游客们为了在游览时能够充分了解景区的人文历史,同时又要能够得到人性化的游览建议,在游览的过程中往往会需要花钱雇专业的导游。导游能熟悉当地环境,熟悉某个景点的人文历史,所以能带领游客参观,能为游客给出专业而且人性化的游览建议。
		
			对于某些非常著名的景点,如泰山等,每年慕名而来的游客数量都十分巨大,过多的人数为游客的游览参观带来了诸多不便。如果此时再雇佣大量的导游,无疑会给游客的游览带来更多不便,使游览体验大打折扣。
		
			另一方面,导游行业也存在其他利益关系,许多商家为了推广商品,往往采取与导游合作的方式,往往逼迫游客进行不必要的消费。同时,游客对于景点当地人文环境的认识不够,也造成了和导游之间的信息不对等,对于导游推荐的内容也难以做出合理的判断。
		
			对于以上这些问题,游客们急需一种能自助使用的导游方式。在游览过程中,游客希望能自助获取自己需要的信息,如景点中相关部分的介绍、游览的建议等。同时,对于一些景点的游览项目,还要能够从其他游览过的游客那里得到信息,要能看到其他游客对于某个游览项目的评论。	
			
		\subsection{研究的目的和意义}	
		
			满足游客的需求,让游客在游玩时可以方便地得到针对景点里某个部分的介绍以及其他游客分享的评价和攻略信息。
  	
  		“游玩伴侣” APP 是一款以替代导游功能为目标的旅游 APP 。用户在游玩时可以拍摄景点某一部分的照片,然后 APP 通过识别照片中的景物来有针对性地告诉游客该景物的相关信息以及其他游客针对该景物的评价和攻略。游客只需拍照上传即可得到所有这些信息,大大减少了游客在游玩时的操作,有效提升游客的游玩体验。
	
	\section{国内外导游软件的现状及分析}
	
		\subsection{国外现状及分析}
	
			TripAdvisor 是全球领先的旅游网站,官方中文名为“猫途鹰”。收录逾5亿条全球旅行者的点评及建议,覆盖超过190个国家的700万个住宿、餐厅和景点,并提供丰富的旅行规划和预订功能。该网站主要收录景点的基本信息以及游客对景点的评价,而且评价数量多。但是该网站仅仅适合在旅游前做准备的阶段使用,可以在旅游前得知某个景点好不好玩和其他一些信息。游客在旅游过程中看到景点的某个部分往往会需要导游的讲解,此时该网站就完全无能为力了。
	
		\subsection{国内现状及分析}
	
			蚂蜂窝旅行网是中国领先的自由行服务平台。蚂蜂窝旅行网由陈罡和吕刚创立于2006年,从2010年正式开始公司化运营。蚂蜂窝的景点、餐饮、酒店等点评信息均来自数千万用户的真实分享,每年帮助过亿的旅行者制定自由行方案。和 TripAdvisor 相同的是,蚂蜂窝也是适合在旅游之前做准备的阶段使用,不能在旅行中途对于景点的某个部分给出相关信息,所以也依然无法有效解决问题。
		
			景点通含有景区地图、路线规划和景点介绍以及语音讲解,同时也提供了景点折扣门票和旅游出行攻略。景点通是一款导游类的 APP ,收录了全国非常多的景点,但是其主要依靠 GPS 定位来识别景区,这样就导致无法方便地针对景点内的某个部分提供导游信息。比如在游玩孔庙的时候,通过 GPS 定位能知道游客在孔庙里,但具体在游玩哪个殿哪个碑却必须要游客手动来选择,十分地麻烦。
	
	\section{国内外图像分类技术的现状及分析}
	
		\subsection{国外现状及分析}
		
			1999年,Chapelle 等将支持向量机( SVM )应用于图片分类问题上;2006年,Nister 等提出了一种基于词典树( VT, Vocabulary Tree )的图像特征表示方法并用于图像分类中,通过使用 K-means 聚类算法对图像 SIFT 关键点聚类生成词典,每个聚类中心构成一个视觉关键字,进而将图像量化为所有视觉关键字的直方图所表示的特征;2007年,Bosch 等将随机树和随机蕨( Random Forests and Ferns )应用于图像分类;2012年, Krizhevsky 等首次将卷积神经网络应用在图像分类领域,所训练的深度卷积神经网络 AlexNet 在比赛中取得了优异成绩。
	
		\subsection{国内现状及分析}
	
			2003年,万华林等在图像分类问题上使用了支持向量机( SVM );2007年,董立岩等将贝叶斯算法应用于图像分类中;2013年,宋相法等提出了基于稀疏编码和集成学习的多示例多标记图像分类法。传统的浅层分类算法的研究已经取得了不错的分类效果,但是只适用于小样本集图像的分类。直到深度学习技术的出现改观了以前小规模机器学习算法的不足。
		
			对于大样本集图像的分类,近来越来越多采用深度学习的方法进行分类。
	
	\section{本文的主要研究内容}
	
		
		
			
	\section{本文结构}
	


\chapter{需求分析}

	\section{需求背景}

		\subsection{当前导游软件市场发展状况}

			当前市场上的旅游类应用多为工具类或者论坛类应用,前者可以为用户提供诸如订购车票、预定酒店、游览计划等等功能,方便用户旅游时的出行体验;后者主要是提供一个能让所有用户自由交流讨论景点的平台,同时也尽可能提供足够多的第一方信息,让用户在游览前能够对想去的景点有足够的了解,还能够帮助部分用户决定将要游览哪个景点。而针对用户游览时的需求的软件则很少,少数的几个也几乎都是需要用户自己选择景点。一方面因为用户在游览时普遍缺乏对景点的了解,有时甚至会不知道景点的名称,所以在使用这些 APP 时体验会大打折扣;另一方面由于这类 APP 多为景区官方提供,所以内容上会比较单一,无法有效地获取其他游客对景点的游览建议。此外,游览时为了获取景点的信息,用户还不得不一直低头看手机,也会十分影响用户的游览体验。

		\subsection{需求的抛出}

			用户需要一个能在游览时使用、并且能方便获取景点信息的导游应用。由于用户可能不熟悉甚至不认识景点,所以这个导游应用需要能够帮助用户了解正在游览的景点的具体信息。同时,用户获取到的信息中,应该包括景区官方提供的信息和游客等第三方提供的信息,增加信息的丰富程度。最后,为了最大程度上不降低用户的游览体验,应用还需要支持类似语音朗读的功能,帮助用户专心游览而不用一直低头看手机。

	\section{解决方案}

		\subsection{需求整理}

			用户对导游类应用的需求主要体现在以下四个方面:

			\begin{enumerate}
				
				\item 基本的工具类功能。用户在游览时会需要一些基本工具类功能的辅助,比如定位、景区地图等,帮助用户在游览时确切了解景区的大小、各景点的位置、用户当前位置等信息,帮助用户更好地制定游览计划。

				\item 景点信息获取。用户往往会对正在游览的景点不够了解,需要通过导游类应用来获取到与景点有关的历史、人文等信息,来获得更好的游览体验。而且这些获取到的信息中应当要足够丰富,既包含景区官方提供的介绍,还应当包括来自其他游客的第三方的建议、评论等,用户能通过这些信息来全方位了解正在游览的景区。

				\item 景点识别辅助。如果用户在游览前没有做足够的了解,那么就会出现不了解甚至是不认识景点的情况。可能用户连正在游览的景点的名字都不知道,这样就很难去有效地获取到景点的相关信息。为此,导游类应用还应当提供针对景点识别方面的辅助,让用户能够很方便地识别出正在游览的景点,进而可以很方便地再去获取到景点的相关信息。

				\item 语音朗读辅助。为了专心游览景点,用户在旅游时会希望能够不需要一直低头看手机屏幕。为此导游类软件应当添加语言朗读的功能,让用户在专心游览的同时还能通过“听”的方式获取到需要的信息,能够大大提升导游类应用的使用体验。

			\end{enumerate}

		\subsection{技术需求}

			根据对用户需求的整理,导游类应用的技术需求主要体现在以下几个方面:

			\begin{enumerate}
				
				\item web 服务器。首先用户需要的信息应当是从网络上获取,这样就能很方便地进行更新,所以将所有信息都存储在一个 web 服务器上是一个很好的选择。用户在获取信息时,通过导游类应用向 web 服务器发送请求,服务器收到请求后再返回用户所需要的数据。现今网络的速度已经非常快,传输大量的文字和图片也无需长时间的等待。同时当用户需要发表游览攻略或者对景区的评论时,只需更新 web 服务器上的数据就可以让所有的用户都能看到新添加的内容。

				\item 地图模块。在游览过程中用户需要景区地图、定位等功能,如果使用现有的地图模块作为景区地图的话就能很好地将景区地图和定位两个功能结合到一起。地图模块本身能包含景区中的部分建筑以及路线,在这个基础上再添加景区中每个景点的位置标注就能形成一个景区的地图,同时地图模块的定位功能也能和这个景区的地图无缝地结合到一起,所以在导游类应用中使用地图模块是一个很好的选择。

				\item 图片识别功能。为了帮助用户识别出正在游览的景点,有两种可能的方案可供选择:定位识别和图片识别。首先来说定位识别,通过卫星定位得到的用户位置信息来判断出用户正在识别的景点,这个方案的优点是即使用户在一些建筑类景点的内部时也可以有效识别,而且速度很快。但是定位识别的缺点也比较明显,一方面在一些景点密集的地方可能会不方便直接通过定位来判断;另一方面当用户并没有真正在游览某个景点,比如只是想知道远处的一个景点时就必须要走过去才能识别成功。而对于图片识别的方案,用户通过拍照或者上传已经拍好的照片来进行景点的识别。优点是用户无需走到景点附近,只需要拍到景点的照片就能进行有效的识别,同时在一些景点密集的地方也能有效使用;缺点则是识别过程复杂,识别速度相对来说比较慢。从方便用户使用的角度考虑的话,图片识别方案将会是一个更好的选择。

				\item 语音朗读功能。语音朗读功能的主要目的是将文本使用语音的方式朗读出来,既然是朗读就需要注意到朗读的语调、语速和停顿等,这些都会影响到用户在使用时的实际体验。目前可以选择使用的方案主要有两种:手机原生支持的 TTS 模块和第三方提供的离线/在线 TTS 方案。首先来说原生的 TTS,因为是系统自带的功能,所以无需再添加第三方的模块,可以有效减小应用的体积;而缺点是功能丰富程度比较有限,对于语调、语速和停顿等的效果相对较一般。而第三方提供的功能一般都会有自己的特色,往往能够更改声音,并且能够提供更好的朗读体验。所以从最终朗读的体验来说,第三方提供的离线/在线 TTS 方案将是一个更好的选择。

			\end{enumerate}
		
	\section{本章小结}
	
		在游览前,用户可以登录论坛浏览其他网友对于景区景点发布的游览攻略和讨论,并与其他网友进行互动,从而满足用户在旅游前选定旅游景区、制定旅游计划的需求。在游览时,用户可以通过地图功能来看到附近景区的具体位置,同时通过景点识别功能可以在游览时方便地识别出景点并获取到景点的具体信息。同时,使用辅助功能的语音朗读,可以边游览景点边听到用户针对该景点而发布的游览攻略,从而满足用户在游览时对导游的需求。


		
\chapter{系统设计}

    \section{系统概要设计}

        因为存在 web 服务器和手机应用两部分的需求,本次课程设计也主要分成两个部分来进行设计,同时 web 服务器还依赖于数据库提供的数据持久化存储的服务。web 服务器负责给手机端的应用提供所有的景区和景点的信息,同时还在 web 服务器上为每个景区维护一个用于图片识别的神经网络模型,为手机端的应用提供图像识别的服务;手机端应用则是用户直接操作的界面,用户可以在手机端应用中完成所有的操作,并通过手机端应用从 web 服务器中获得所有的信息和完成景点识别的功能。

    \section{功能结构设计}

        功能结构设计如图 3-1 所示,主要分为 WEB 服务器和 APP 客户端两大部分:

        \begin{figure}[htbp]
            \includegraphics[width=15cm]{structure.pdf}
            \centering
            \caption{功能结构图}
        \end{figure}

        \subsection{WEB 服务器功能结构设计}

        \begin{enumerate}
            
            \item 用户验证。用户在发表攻略、发表讨论、点赞或者评论之前 WEB 服务器需要向用户验证当前身份,以确保用户是在登录状态下才能继续完成上述操作。每个内容都需要绑定至一个特定的用户,此时该用户作为该内容的作者。

            \item 内容检索。用户在通过 APP 客户端获取特定的内容时,APP 客户端会帮助用户向 WEB 服务器发送一个请求,然后 WEB 服务器按照用户的要求去检索已有的数据,并将数据打包成一个 json 对象返回给 APP 客户端。

            \item 数据存储。WEB 服务器还需要连接到一个部署在本地的数据库,用来做数据的持久化存储。WEB 服务器能检索到的所有数据都来自于这个数据库。

            \item 景点识别。景点识别需要单独作为一个外部的功能调用来实现,可以将训练好的卷积神经网络模型提前加载至内存中,在需要识别的时候直接使用这个训练好的神经网络来识别图像。

        \end{enumerate}

        \subsection{APP 客户端功能结构设计}

        \begin{enumerate}
            
            \item 用户注册。用户在初次使用时需要向 WEB 服务器注册自己,而这个功能需要通过 APP 客户端的界面来帮助实现。用户在使用时可以直接在 APP 客户端的界面中进行注册和登录的操作。
            
            \item 景区地图。用户在游览时需要的景区地图直接使用接入的地图模块来实现,在用户选择景区之后将地图中心定位至景区所在位置,并放大地图至合适的比例让用户得以通过地图大致浏览当前景区。
            
            \item 景点定位。在选择完景区之后,该景区内的所有景点的位置都会在地图上标出,配和景区地图来帮助用户看到自己的位置和景点位置的相对关系。
            
            \item 景点评分。用户可以对游览过的景点进行评分,评分过后在景点的首页上会显示出所有用户对该景点的平均评分,评分以星级的方式显示出来。
            
            \item 上传图片。用户需要识别图片的时候,可以通过 APP 客户端拍照或者是直接从相册中选择已经提前拍好的图片。APP 客户端将图片上传至 WEB 服务器,之后服务器将该图片用于景点识别并将最后的识别结果返回给 APP 客户端。
            
            \item 内容发布。服务器中除景区和景点相关信息之外的内容都是由用户们发布的,用户在 APP 客户端中简单点击一下按钮就能进入到内容发布的界面,发布成功的话就能在 WEB 服务器对应的数据库中添加内容。
            
            \item 内容浏览。在 APP 客户端向 WEB 服务器发送请求并接收到返回的数据之后,APP 客户端再对接收到的数据进行渲染,最后将渲染好的内容在界面中呈现给用户。
            
            \item 点赞评论。对于喜欢的评论,用户可以使用点赞的方式支持,点赞的人数代表着人气并且在一定程度上也代表着内容的质量;如果用户对该内容深有感触或者持有不同观点,也可以通过评论的方式来表达。
            
            \item 语音朗读。语音朗读能将指定的文字转换为语音朗读给用户,在用户查看内容时能解放双眼,直接听到想要了解的内容。

        \end{enumerate}

    % \section{系统流程设计}

    \section{数据库设计}

        \subsection{数据库总体设计及含义}
            数据库部分总体设计即所有表的含义如下:

            \begin{enumerate}
                
                \item 景区表(scenic\_area),存储一个景区的数据;
                \item 景点表(scenic\_spot)则是存储了景点的数据,每条景点数据都必须要有一个非空的外键指向一条景区表的数据;
                \item 用户表(user)存储了用户的数据,包括了用户的用户名、密码、邮箱以及是否是管理员等;
                \item 攻略表(article)存储用户发布的攻略的数据,每篇攻略都有一个非空外键指向用户表,作为该攻略的作者;
                \item 讨论表(discussion)存储用户发布的讨论的数据,每篇讨论都有一个非空外键指向用户表,作为该讨论的作者;
                \item 评论表(comment)存储用户发布的评论的数据,每篇评论都有一个非空外键指向用户表,作为该评论的作者;
                \item 景区攻略表(area\_article)存储景区和攻略的关系,一个非空的外键指向景区表,另一个非空的外键指向攻略表,每一条景区攻略表的记录代表该攻略是针对对应的景区的;
                \item 景区讨论表(area\_discussion)存储景区和讨论的关系,一个非空的外键指向景区表,另一个非空的外键指向讨论表,每一条景区讨论表的记录代表该讨论是针对对应的景区的;
                \item 景点攻略表(spot\_article)存储景点和攻略的关系,一个非空的外键指向景点表,另一个非空的外键指向攻略表,每一条景点攻略表的记录代表该攻略是针对对应的景点的;
                \item 景点讨论表(spot\_discussion)存储景点和讨论的关系,一个非空的外键指向景点表,另一个非空的外键指向讨论表,每一条景点讨论表的记录代表该讨论是针对对应的景点的;
                \item 景区评分表(area\_score)存储用户对景区的评分,一个非空的外键指向用户表,另一个非空的外键指向景区表,并且两个外键不能同时重复;
                \item 景点评分表(spot\_score)存储用户对景点的评分,一个非空的外键指向用户表,另一个非空的外键指向景点表,并且两个外键不能同时重复;
                \item 攻略点赞表(vote\_article)存储用户对攻略的点赞,一个非空的外键指向用户表,另一个非空的外键指向攻略表,表示该用户对该攻略点了赞;
                \item 讨论点赞表(vote\_discussion)存储用户对讨论的点赞,一个非空的外键指向用户表,另一个非空的外键指向讨论表,表示该用户对该讨论点了赞;
                \item 攻略评论表(article\_comment)存储对攻略的评论,一个非空的外键指向攻略表,另一个非空的外键指向评论表,表示该评论是对应该攻略的评论;
                \item 讨论评论表(discussion\_comment)存储对讨论的评论,一个非空的外键指向讨论表,另一个非空的外键指向评论表,表示该评论时对应该讨论的评论;
                \item 评论评论表(comment\_comment)存储对评论的评论,两个非空的外键指向评论表,一个做为源评论,一个作为目标评论,表示源评论是对目标评论的评论。

            \end{enumerate}

        \subsection{数据库具体设计及字段说明}

            下面详细列出每个表的具体字段和说明:

            \begin{table}[htbp]
                \caption{\wuhao 景区表(scenic\_area)}
                \vspace{0.5em}\centering\wuhao
                \begin{tabular}{llll}
                    \toprule[1.5pt]
                    字段 & 类型 & 限制 & 说明\\
                    \midrule[1pt]
                    id & int(11) & PK & 景区的ID\\
                    name & varchar(200) & 非空 & 景区的名称\\
                    latitude & double & 非空 & 景区的经度\\
                    longitude & double & 非空 & 景区的纬度\\
                    about & longtext & 无 & 景区的介绍文本\\
                    image & varchar(200) & 无 & 景区介绍图片的路径\\
                    \bottomrule[1.5pt]
                \end{tabular}
            \end{table}

            \begin{table}[htbp]
                \caption{\wuhao 景点表(scenic\_spot)}
                \vspace{0.5em}\centering\wuhao
                \begin{tabular}{llll}
                    \toprule[1.5pt]
                    字段 & 类型 & 限制 & 说明\\
                    \midrule[1pt]
                    id & int(11) & AI PK & 景点的ID\\
                    name & varchar(200) & 非空 & 景点的名称\\
                    latitude & double & 非空 & 景点的经度\\
                    longitude & double & 非空 & 景点的纬度\\
                    about & longtext & 无 & 景点的介绍文本\\
                    image & varchar(200) & 无 & 景点介绍图片的路径\\
                    area\_id & int(11) & 非空 & 景点所在景区的ID\\
                    \bottomrule[1.5pt]
                \end{tabular}
            \end{table}

            \begin{table}[htbp]
                \caption{\wuhao 用户表(user)}
                \vspace{0.5em}\centering\wuhao
                \begin{tabular}{llll}
                    \toprule[1.5pt]
                    字段 & 类型 & 限制 & 说明\\
                    \midrule[1pt]
                    id & int(11) & PK & 用户的ID\\
                    username & varchar(150) & 非空 & 用户名\\
                    password & varchar(128) & 非空 & 用户密码\\
                    email & varchar(254) & 无 & 用户的邮件\\
                    is\_superuser & tinyint(1) & 非空 & 用户是否为管理员\\ 
                    \bottomrule[1.5pt]
                \end{tabular}
            \end{table}

            \begin{table}[htbp]
                \caption{\wuhao 攻略表(article)}
                \vspace{0.5em}\centering\wuhao
                \begin{tabular}{llll}
                    \toprule[1.5pt]
                    字段 & 类型 & 限制 & 说明\\
                    \midrule[1pt]
                    id & int(11) & PK & 攻略的ID\\
                    title & varchar(200) & 非空 & 攻略的标题\\
                    content & longtext & 非空 & 攻略的内容\\
                    user\_id & int(11) & 非空 & 用户的ID\\
                    \bottomrule[1.5pt]
                \end{tabular}
            \end{table}

            \begin{table}[htbp]
                \caption{\wuhao 讨论表(discussion)}
                \vspace{0.5em}\centering\wuhao
                \begin{tabular}{llll}
                    \toprule[1.5pt]
                    字段 & 类型 & 限制 & 说明\\
                    \midrule[1pt]
                    id & int(11) & PK & 讨论的ID\\
                    content & longtext & 非空 & 讨论的内容\\
                    user\_id & int(11) & 非空 & 用户的ID\\
                    \bottomrule[1.5pt]
                \end{tabular}
            \end{table}

            \begin{table}[htbp]
                \caption{\wuhao 评论表(comment)}
                \vspace{0.5em}\centering\wuhao
                \begin{tabular}{llll}
                    \toprule[1.5pt]
                    字段 & 类型 & 限制 & 说明\\
                    \midrule[1pt]
                    id & int(11) & PK & 评论的ID\\
                    content & longtext & 非空 & 评论的内容\\
                    user\_id & int(11) & 非空 & 用户的ID\\
                    \bottomrule[1.5pt]
                \end{tabular}
            \end{table}

            \begin{table}[htbp]
                \caption{\wuhao 景区攻略表(area\_article)}
                \vspace{0.5em}\centering\wuhao
                \begin{tabular}{llll}
                    \toprule[1.5pt]
                    字段 & 类型 & 限制 & 说明\\
                    \midrule[1pt]
                    id & int(11) & PK & ID\\
                    article\_id & int(11) & 非空 & 攻略的ID\\
                    area\_id & int(11) & 非空 & 景区的ID\\
                    \bottomrule[1.5pt]
                \end{tabular}
            \end{table}

            \begin{table}[htbp]
                \caption{\wuhao 景区讨论表(area\_discussion)}
                \vspace{0.5em}\centering\wuhao
                \begin{tabular}{llll}
                    \toprule[1.5pt]
                    字段 & 类型 & 限制 & 说明\\
                    \midrule[1pt]
                    id & int(11) & PK & ID\\
                    discussion\_id & int(11) & 非空 & 讨论的ID\\
                    area\_id & int(11) & 非空 & 景区的ID\\
                    \bottomrule[1.5pt]
                \end{tabular}
            \end{table}

            \begin{table}[htbp]
                \caption{\wuhao 景点攻略表(spot\_article)}
                \vspace{0.5em}\centering\wuhao
                \begin{tabular}{llll}
                    \toprule[1.5pt]
                    字段 & 类型 & 限制 & 说明\\
                    \midrule[1pt]
                    id & int(11) & PK & ID\\
                    article\_id & int(11) & 非空 & 攻略的ID\\
                    spot\_id & int(11) & 非空 & 景点的ID\\
                    \bottomrule[1.5pt]
                \end{tabular}
            \end{table}

            \begin{table}[htbp]
                \caption{\wuhao 景点讨论表(spot\_discussion)}
                \vspace{0.5em}\centering\wuhao
                \begin{tabular}{llll}
                    \toprule[1.5pt]
                    字段 & 类型 & 限制 & 说明\\
                    \midrule[1pt]
                    id & int(11) & PK & ID\\
                    discussion\_id & int(11) & 非空 & 讨论的ID\\
                    spot\_id & int(11) & 非空 & 景点的ID\\
                    \bottomrule[1.5pt]
                \end{tabular}
            \end{table}

            \begin{table}[htbp]
                \caption{\wuhao 景区评分表(area\_score)}
                \vspace{0.5em}\centering\wuhao
                \begin{tabular}{llll}
                    \toprule[1.5pt]
                    字段 & 类型 & 限制 & 说明\\
                    \midrule[1pt]
                    id & int(11) & PK & ID\\
                    score & int(11) & 非空 & 评分\\
                    area\_id & int(11) & 非空 & 景区的ID\\
                    user\_id & int(11) & 非空 & 用户的ID\\
                    \bottomrule[1.5pt]
                \end{tabular}
            \end{table}

            \begin{table}[htbp]
                \caption{\wuhao 景点评分表(spot\_score)}
                \vspace{0.5em}\centering\wuhao
                \begin{tabular}{llll}
                    \toprule[1.5pt]
                    字段 & 类型 & 限制 & 说明\\
                    \midrule[1pt]
                    id & int(11) & PK & ID\\
                    score & int(11) & 非空 & 评分\\
                    spot\_id & int(11) & 非空 & 景点的ID\\
                    user\_id & int(11) & 非空 & 用户的ID\\
                    \bottomrule[1.5pt]
                \end{tabular}
            \end{table}

            \begin{table}[htbp]
                \caption{\wuhao 攻略点赞表(vote\_article)}
                \vspace{0.5em}\centering\wuhao
                \begin{tabular}{llll}
                    \toprule[1.5pt]
                    字段 & 类型 & 限制 & 说明\\
                    \midrule[1pt]
                    id & int(11) & PK & ID\\
                    article\_id & int(11) & 非空 & 攻略的ID\\
                    user\_id & int(11) & 非空 & 用户的ID\\
                    \bottomrule[1.5pt]
                \end{tabular}
            \end{table}

            \begin{table}[htbp]
                \caption{\wuhao 讨论点赞表(vote\_discussion)}
                \vspace{0.5em}\centering\wuhao
                \begin{tabular}{llll}
                    \toprule[1.5pt]
                    字段 & 类型 & 限制 & 说明\\
                    \midrule[1pt]
                    id & int(11) & PK & ID\\
                    discussion\_id & int(11) & 非空 & 讨论的ID\\
                    user\_id & int(11) & 非空 & 用户的ID\\
                    \bottomrule[1.5pt]
                \end{tabular}
            \end{table}

            \begin{table}[htbp]
                \caption{\wuhao 攻略评论表(article\_comment)}
                \vspace{0.5em}\centering\wuhao
                \begin{tabular}{llll}
                    \toprule[1.5pt]
                    字段 & 类型 & 限制 & 说明\\
                    \midrule[1pt]
                    id & int(11) & PK & ID\\
                    article\_id & int(11) & 非空 & 攻略的ID\\
                    comment\_id & int(11) & 非空 & 评论的ID\\
                    \bottomrule[1.5pt]
                \end{tabular}
            \end{table}

            \begin{table}[htbp]
                \caption{\wuhao 讨论评论表(discussion\_comment)}
                \vspace{0.5em}\centering\wuhao
                \begin{tabular}{llll}
                    \toprule[1.5pt]
                    字段 & 类型 & 限制 & 说明\\
                    \midrule[1pt]
                    id & int(11) & PK & ID\\
                    discussion\_id & int(11) & 非空 & 讨论的ID\\
                    comment\_id & int(11) & 非空 & 评论的ID\\
                    \bottomrule[1.5pt]
                \end{tabular}
            \end{table}

            \begin{table}[htbp]
                \caption{\wuhao 评论评论表(comment\_comment)}
                \vspace{0.5em}\centering\wuhao
                \begin{tabular}{llll}
                    \toprule[1.5pt]
                    字段 & 类型 & 限制 & 说明\\
                    \midrule[1pt]
                    id & int(11) & PK & ID\\
                    comment\_d\_id & int(11) & 非空 & 目标评论的ID\\
                    comment\_s\_id & int(11) & 非空 & 源评论的ID\\
                    \bottomrule[1.5pt]
                \end{tabular}
            \end{table}

    \section{本章小结}

        本章主要讨论了系统功能结构和数据库的详细设计。系统功能结构方面首先通过一个整体的系统功能结构图给出了总体的设计以及功能模块的划分,再对数据库的各个表进行设计,最后对每个表内的每个字段进行设计,以完成整个系统的功能设计。
%\chapter{图像识别模块实现与测试}

    \section{图像识别模块的实现}

    \section{图像识别模块的测试}
\chapter{系统实现与测试}

    \section{title}

        

\backmatter
%硕博书序
\include{back/conclusion}   % 结论
\bibliographystyle{hithesis} %如果没有参考文献时候
\bibliography{reference}
% \begin{appendix}%附录
% \input{back/appA.tex}
% \end{appendix}
% \include{back/publications}    % 所发文章
% \include{back/ceindex}    % 索引, 根据自己的情况添加或者不添加,选择自动添加或者手工添加。
\authorization %授权
%\authorization[saomiao.pdf] %添加扫描页的命令,与上互斥
\include{back/acknowledgements} %致谢
% \include{back/resume}          % 博士学位论文有个人简介

%本科书序为:
%\include{body/conclusion}   % 结论
%\bibliographystyle{hithesis}
%\bibliography{reference}
%\authorization %授权
%%\authorization[saomiao.pdf] %添加扫描页的命令,与上互斥
%\include{body/acknowledgements} %致谢
%\begin{appendix}%附录
%\input{body/appendix01}%本科生翻译论文
%\end{appendix}

\end{document}
