\chapter{绪论}
	
	\section{课题背景及研究的目的和意义}
	
		\subsection{课题背景}
		
			随着中国经济的发展,人们的消费水平越来越高,旅游越来越成为一种普遍的爱好。每逢节假日,各地景区都会迎来许多游客。游客们为了在游览时能够充分了解景区的人文历史,同时又要能够得到人性化的游览建议,在游览的过程中往往会需要花钱雇专业的导游。导游能熟悉当地环境,熟悉某个景点的人文历史,所以能带领游客参观,能为游客给出专业而且人性化的游览建议。
		
			对于某些非常著名的景点,如泰山等,每年慕名而来的游客数量都十分巨大,过多的人数为游客的游览参观带来了诸多不便。如果此时再雇佣大量的导游,无疑会给游客的游览带来更多不便,使游览体验大打折扣。
		
			另一方面,导游行业也存在其他利益关系,许多商家为了推广商品,往往采取与导游合作的方式,往往逼迫游客进行不必要的消费。同时,游客对于景点当地人文环境的认识不够,也造成了和导游之间的信息不对等,对于导游推荐的内容也难以做出合理的判断。
		
			对于以上这些问题,游客们急需一种能自助使用的导游方式。在游览过程中,游客希望能自助获取自己需要的信息,如景点中相关部分的介绍、游览的建议等。同时,对于一些景点的游览项目,还要能够从其他游览过的游客那里得到信息,要能看到其他游客对于某个游览项目的评论。	
			
		\subsection{研究的目的和意义}	
		
			满足游客的需求,让游客在游玩时可以方便地得到针对景点里某个部分的介绍以及其他游客分享的评价和攻略信息。
  	
  		“游玩伴侣” APP 是一款以替代导游功能为目标的旅游 APP 。用户在游玩时可以拍摄景点某一部分的照片,然后 APP 通过识别照片中的景物来有针对性地告诉游客该景物的相关信息以及其他游客针对该景物的评价和攻略。游客只需拍照上传即可得到所有这些信息,大大减少了游客在游玩时的操作,有效提升游客的游玩体验。
	
	\section{国内外导游软件的现状及分析}
	
		\subsection{国外现状及分析}
	
			TripAdvisor 是全球领先的旅游网站,官方中文名为“猫途鹰”。收录逾5亿条全球旅行者的点评及建议,覆盖超过190个国家的700万个住宿、餐厅和景点,并提供丰富的旅行规划和预订功能。该网站主要收录景点的基本信息以及游客对景点的评价,而且评价数量多。但是该网站仅仅适合在旅游前做准备的阶段使用,可以在旅游前得知某个景点好不好玩和其他一些信息。游客在旅游过程中看到景点的某个部分往往会需要导游的讲解,此时该网站就完全无能为力了。
	
		\subsection{国内现状及分析}
	
			蚂蜂窝旅行网是中国领先的自由行服务平台。蚂蜂窝旅行网由陈罡和吕刚创立于2006年,从2010年正式开始公司化运营。蚂蜂窝的景点、餐饮、酒店等点评信息均来自数千万用户的真实分享,每年帮助过亿的旅行者制定自由行方案。和 TripAdvisor 相同的是,蚂蜂窝也是适合在旅游之前做准备的阶段使用,不能在旅行中途对于景点的某个部分给出相关信息,所以也依然无法有效解决问题。
		
			景点通含有景区地图、路线规划和景点介绍以及语音讲解,同时也提供了景点折扣门票和旅游出行攻略。景点通是一款导游类的 APP ,收录了全国非常多的景点,但是其主要依靠 GPS 定位来识别景区,这样就导致无法方便地针对景点内的某个部分提供导游信息。比如在游玩孔庙的时候,通过 GPS 定位能知道游客在孔庙里,但具体在游玩哪个殿哪个碑却必须要游客手动来选择,十分地麻烦。
	
	\section{国内外图像分类技术的现状及分析}
	
		\subsection{国外现状及分析}
		
			1999年,Chapelle 等将支持向量机( SVM )应用于图片分类问题上;2006年,Nister 等提出了一种基于词典树( VT, Vocabulary Tree )的图像特征表示方法并用于图像分类中,通过使用 K-means 聚类算法对图像 SIFT 关键点聚类生成词典,每个聚类中心构成一个视觉关键字,进而将图像量化为所有视觉关键字的直方图所表示的特征;2007年,Bosch 等将随机树和随机蕨( Random Forests and Ferns )应用于图像分类;2012年, Krizhevsky 等首次将卷积神经网络应用在图像分类领域,所训练的深度卷积神经网络 AlexNet 在比赛中取得了优异成绩。
	
		\subsection{国内现状及分析}
	
			2003年,万华林等在图像分类问题上使用了支持向量机( SVM );2007年,董立岩等将贝叶斯算法应用于图像分类中;2013年,宋相法等提出了基于稀疏编码和集成学习的多示例多标记图像分类法。传统的浅层分类算法的研究已经取得了不错的分类效果,但是只适用于小样本集图像的分类。直到深度学习技术的出现改观了以前小规模机器学习算法的不足。
		
			对于大样本集图像的分类,近来越来越多采用深度学习的方法进行分类。
	
	\section{本文的主要研究内容}
	
		
		
			
	\section{本文结构}
	

