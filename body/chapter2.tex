\chapter{需求分析}

	\section{需求背景}

		\subsection{当前导游软件市场发展状况}

			当前市场上的旅游类应用多为工具类或者论坛类应用,前者可以为用户提供诸如订购车票、预定酒店、游览计划等等功能,方便用户旅游时的出行体验;后者主要是提供一个能让所有用户自由交流讨论景点的平台,同时也尽可能提供足够多的第一方信息,让用户在游览前能够对想去的景点有足够的了解,还能够帮助部分用户决定将要游览哪个景点。而针对用户游览时的需求的软件则很少,少数的几个也几乎都是需要用户自己选择景点。一方面因为用户在游览时普遍缺乏对景点的了解,有时甚至会不知道景点的名称,所以在使用这些 APP 时体验会大打折扣;另一方面由于这类 APP 多为景区官方提供,所以内容上会比较单一,无法有效地获取其他游客对景点的游览建议。此外,游览时为了获取景点的信息,用户还不得不一直低头看手机,也会十分影响用户的游览体验。

		\subsection{需求的抛出}

			用户需要一个能在游览时使用、并且能方便获取景点信息的导游应用。由于用户可能不熟悉甚至不认识景点,所以这个导游应用需要能够帮助用户了解正在游览的景点的具体信息。同时,用户获取到的信息中,应该包括景区官方提供的信息和游客等第三方提供的信息,增加信息的丰富程度。最后,为了最大程度上不降低用户的游览体验,应用还需要支持类似语音朗读的功能,帮助用户专心游览而不用一直低头看手机。

	\section{解决方案}

		\subsection{需求整理}

			用户对导游类应用的需求主要体现在以下四个方面:

			\begin{enumerate}
				
				\item 基本的工具类功能。用户在游览时会需要一些基本工具类功能的辅助,比如定位、景区地图等,帮助用户在游览时确切了解景区的大小、各景点的位置、用户当前位置等信息,帮助用户更好地制定游览计划。

				\item 景点信息获取。用户往往会对正在游览的景点不够了解,需要通过导游类应用来获取到与景点有关的历史、人文等信息,来获得更好的游览体验。而且这些获取到的信息中应当要足够丰富,既包含景区官方提供的介绍,还应当包括来自其他游客的第三方的建议、评论等,用户能通过这些信息来全方位了解正在游览的景区。

				\item 景点识别辅助。如果用户在游览前没有做足够的了解,那么就会出现不了解甚至是不认识景点的情况。可能用户连正在游览的景点的名字都不知道,这样就很难去有效地获取到景点的相关信息。为此,导游类应用还应当提供针对景点识别方面的辅助,让用户能够很方便地识别出正在游览的景点,进而可以很方便地再去获取到景点的相关信息。

				\item 语音朗读辅助。为了专心游览景点,用户在旅游时会希望能够不需要一直低头看手机屏幕。为此导游类软件应当添加语言朗读的功能,让用户在专心游览的同时还能通过“听”的方式获取到需要的信息,能够大大提升导游类应用的使用体验。

			\end{enumerate}

		\subsection{技术需求}

			根据对用户需求的整理,导游类应用的技术需求主要体现在以下几个方面:

			\begin{enumerate}
				
				\item web 服务器。首先用户需要的信息应当是从网络上获取,这样就能很方便地进行更新,所以将所有信息都存储在一个 web 服务器上是一个很好的选择。用户在获取信息时,通过导游类应用向 web 服务器发送请求,服务器收到请求后再返回用户所需要的数据。现今网络的速度已经非常快,传输大量的文字和图片也无需长时间的等待。同时当用户需要发表游览攻略或者对景区的评论时,只需更新 web 服务器上的数据就可以让所有的用户都能看到新添加的内容。

				\item 地图模块。在游览过程中用户需要景区地图、定位等功能,如果使用现有的地图模块作为景区地图的话就能很好地将景区地图和定位两个功能结合到一起。地图模块本身能包含景区中的部分建筑以及路线,在这个基础上再添加景区中每个景点的位置标注就能形成一个景区的地图,同时地图模块的定位功能也能和这个景区的地图无缝地结合到一起,所以在导游类应用中使用地图模块是一个很好的选择。

				\item 图片识别功能。为了帮助用户识别出正在游览的景点,有两种可能的方案可供选择:定位识别和图片识别。首先来说定位识别,通过卫星定位得到的用户位置信息来判断出用户正在识别的景点,这个方案的优点是即使用户在一些建筑类景点的内部时也可以有效识别,而且速度很快。但是定位识别的缺点也比较明显,一方面在一些景点密集的地方可能会不方便直接通过定位来判断;另一方面当用户并没有真正在游览某个景点,比如只是想知道远处的一个景点时就必须要走过去才能识别成功。而对于图片识别的方案,用户通过拍照或者上传已经拍好的照片来进行景点的识别。优点是用户无需走到景点附近,只需要拍到景点的照片就能进行有效的识别,同时在一些景点密集的地方也能有效使用;缺点则是识别过程复杂,识别速度相对来说比较慢。从方便用户使用的角度考虑的话,图片识别方案将会是一个更好的选择。

				\item 语音朗读功能。语音朗读功能的主要目的是将文本使用语音的方式朗读出来,既然是朗读就需要注意到朗读的语调、语速和停顿等,这些都会影响到用户在使用时的实际体验。目前可以选择使用的方案主要有两种:手机原生支持的 TTS 模块和第三方提供的离线/在线 TTS 方案。首先来说原生的 TTS,因为是系统自带的功能,所以无需再添加第三方的模块,可以有效减小应用的体积;而缺点是功能丰富程度比较有限,对于语调、语速和停顿等的效果相对较一般。而第三方提供的功能一般都会有自己的特色,往往能够更改声音,并且能够提供更好的朗读体验。所以从最终朗读的体验来说,第三方提供的离线/在线 TTS 方案将是一个更好的选择。

			\end{enumerate}
		
	\section{本章小结}
	
		在游览前,用户可以登录论坛浏览其他网友对于景区景点发布的游览攻略和讨论,并与其他网友进行互动,从而满足用户在旅游前选定旅游景区、制定旅游计划的需求。在游览时,用户可以通过地图功能来看到附近景区的具体位置,同时通过景点识别功能可以在游览时方便地识别出景点并获取到景点的具体信息。同时,使用辅助功能的语音朗读,可以边游览景点边听到用户针对该景点而发布的游览攻略,从而满足用户在游览时对导游的需求。


		